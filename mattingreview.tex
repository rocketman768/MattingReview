\documentclass{beamer}

\usepackage[utf8]{inputenc}
\usepackage{default}

\title{Natural Image Matting}
\author{Philip Greggory Lee}
\institute{
 Electrical Engineering and Computer Science\\
 Northwestern University\\
 Evanston, IL 60208
}

\begin{document}

\begin{frame}
 \titlepage
\end{frame}

\section{Introduction}%========================================================

\begin{frame}{The Problem}
 \begin{itemize}
  \item \textit{Alpha matting} is extracting an object from an image.
  \item Each pixel $i$ is a mixture of foreground $F$ and background $B$ layers.
 \end{itemize}
 \begin{align}
   I_i = \alpha_i F_i + (1-\alpha_i)B_i.
 \end{align}
 \begin{itemize}
  \item $\alpha \in [0,1]$ is the mixing layer. Want to recover it.
  \item 3 unknowns to estimate per pixel: severely underconstrained.
 \end{itemize}
\end{frame}

\begin{frame}{What I Have Learned}
 I have spent a very long time thinking about matting. Let me point out the
 most important things I have learned:
 \begin{itemize}
  \item It is an ill-defined problem. No one can define ``foreground'' until
        you have a specific purpose.
  \item No one can \textit{really} solve it. The best they can do is build a
        model of foreground and background appearance and tell you the
        probability that a pixel belongs to foreground or background.
  \item Every algorithm can perform very well when the user constraints are
        dense. Everyone relies on this.
 \end{itemize}
\end{frame}

\section{Matting by Laplacian}%================================================

\begin{frame}{Poisson Matting \cite{sun2004poisson}}
 
\end{frame}

\begin{frame}{Closed-Form Matting \cite{levin2008closed}}
 
\end{frame}

\begin{frame}{Learning-Based Matting \cite{zheng-learning}}
 
\end{frame}

\begin{frame}{Fast Matting \cite{he2010fast}}
 
\end{frame}


% References ==================================================================
\begin{frame}[allowframebreaks]
 \frametitle{References}
 \bibliographystyle{amsalpha}
 \bibliography{mattingrefs.bib}
\end{frame}

\end{document}
